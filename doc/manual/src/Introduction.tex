% !TEX root = EUDAQUserManual.tex
\section{Introduction}
The EUDAQ software is a data acquisition framework, written in C++,
and designed to be modular and portable, running on Linux, Mac OS X, and Windows.
It was written primarily to run the EUDET-type beam telescope~\cite{Roloff:2009zza,Jansen:2016},
but is designed to be generally useful for other systems.

The hardware-specific parts are kept separate from the core,
so that the core library can still be used independently.
For example, hardware-specific parts are two components for the EUDET-type beam telescope: \gls{TLU} and \gls{NI} for Mimosa 26 sensor read out.

The raw data files generated by the DAQ can be converted to the \gls{LCIO} format,
allowing for analysing the data using the EUTelescope package \cite{eutel2008}.

\subsection{Architecture}
It is split into a number of different processes,
each communicating using TCP/IP sockets (compare \autoref{fig:DAQ}).
A central Run Control provides an interface for controlling the whole DAQ system;
other processes connect to the Run Control to receive commands and to report their status.

\begin{figure}[htb]
  \begin{center}
    \includegraphics[width=0.9\textwidth]{src/images/eudaq_working_principle}
    \caption{Schematic of the EUDAQ architecture \cite{Spannagel:2016}.}
    \label{fig:DAQ}
  \end{center}
\end{figure}

Each hardware that produces data (e.g. the \gls{TLU}, the \gls{NI}, or a \gls{DUT}) will have a Producer process (on the left in \autoref{fig:DAQ}).
This will initialize, configure, stop and start the hardware by receiving the commands from the Run Control (red arrows), read out the data and send it to the Data Collector (blue arrows).

The Data Collector receives all the data streams from all the Producers,
and combines them into a single stream that is written to disk (Storage).
It writes the data in a native raw binary format,
but it can be configured to write in other formats, such as \gls{LCIO}.

The Log Collector receives log messages from all other processes (grey arrows),
and displays them to the user, as well as writing them all to file.
This allows for easier debugging, since all log messages are stored together in a central location.

The Monitor reads the data file and generates online-monitoring plots for display.
In the schematic it is shown to communicate with the Data Collector via a socket,
but it actually just reads the data file from disk.

\subsection{Directory and File Structure}
The EUDAQ software is split into several parts that can each be compiled independently,
and are kept in separate subdirectories.
The general structure is outlined below:

\begin{myitemize}
\item \texttt{main}
  contains the core EUDAQ library with the parts that are common to most of the software,
  and several command-line programs that depend only on this library.
  All definitions in the library should be inside the \texttt{eudaq} namespace.
  It is organised into the following subdirectories:
  \begin{myitemize}
  \item \texttt{lib/src}
    contains the library source code,
  \item \texttt{exe/src}
    contains the (command line) executables source code,
  \item \texttt{include}
    contains the header files inside the \texttt{eudaq} subdirectory (to match the namespace),
  \end{myitemize}
\item \texttt{gui}
  contains the graphical programs that are built with Qt, such as the RunControl and LogCollector.
\item \texttt{producers}
  contains all (user-provided) producers shipped with the EUDAQ
  distribution, for example:
  \begin{myitemize}
\item \texttt{tlu}
  contain the parts that depend on the \gls{TLU}.
\item \texttt{ni}
  contain the parts that depend on the \gls{NI} system for Mimosa 26 read out.
\item e.g. \texttt{depfet}, \texttt{fortis}, \texttt{taki}\ldots{}
  contain the code for third-party producers that have been used with
  EUDET-type beam telescopes.
  \end{myitemize}
\item \texttt{extern}
  stores external software that is not part of EUDAQ itself, but that is needed by EUDAQ in some cases,
  such as the \texttt{ZestSC1} driver and the \texttt{tlufirmware} for the \gls{TLU}.
% and the Tsi148 VME driver.
\item \texttt{bin} and \texttt{lib}
  contain the compiled binaries (executables and libraries) generated from the other directories.
\item \texttt{conf}
  contains configuration files for running the beam telescope.
\item \texttt{data} and \texttt{logs}
  are directories for storing the data and log files generated while running the DAQ.
\item \texttt{doc}
  contains documentation, such as this manual.
\end{myitemize}

Each directory containing code has its own \texttt{src} and \texttt{include} subdirectories,
as well as a local \texttt{CMakeLists.txt} file containing the rules
for building that directory using \texttt{CMake}.
Header files usually have a \texttt{.hh} extension so that they can be automatically recognised as C++
(as opposed to C), and source files have either \texttt{.cc} for parts of a library or \texttt{.cxx} for executables.

Each directory can contain a \texttt{README.md} file for brief documentation for this specific part, e.g.  
as installation advice. 
Using the \texttt{*.md} file ending allows for applying the Markdown language \cite{markdownWWW}. 
Accordingly, content will be formatted on the the GitHub platform, where the code is hosted online.
